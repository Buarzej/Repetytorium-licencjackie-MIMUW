\usepackage[utf8]{inputenc}
\usepackage[T1]{fontenc}
\usepackage{polski}
\usepackage[margin = 2.5cm]{geometry}
\usepackage{anyfontsize}
\usepackage{amsmath}
\usepackage{amssymb}
\usepackage{amsfonts}
\usepackage{amsthm}
\usepackage{dsfont}
\usepackage[shortlabels]{enumitem}
\usepackage{mathdots}
\usepackage{array}
\usepackage{enumitem}
\usepackage{hyperref} % referencje do miejsc w pdfie
\usepackage{url}
\usepackage{xfrac} % ukośne ułamki
\usepackage{cancel} % skracanie
\usepackage{environ} % zaawansowane środowiska
\usepackage{minted} % formatowanie kodu
\usepackage{tabularx} % lepsze tabelki
\usepackage{wrapfig} % do ustawiania pozycji obrazków
\usepackage{multicol} % wiele kolumn
\usepackage{bussproofs} % logika

% --- Podstawowe matematyczne symbole
\newcommand{\wtw}{\quad \Leftrightarrow \quad}  % wtedy i tylko wtedy
\newcommand{\bigexists}{\mbox{\Large $\mathsurround0pt\exists$}\hspace{0.2em}} 
\newcommand{\bigforall}{\mbox{\Large $\mathsurround0pt\forall$}\hspace{0.1em}}
\newcommand{\lr}[1]{\left\langle #1 \right\rangle}

% --- Oznaczenia zbiorów i grup
\newcommand{\RR}{\mathbb{R}}
\newcommand{\ZZ}{\mathbb{Z}}
\newcommand{\NN}{\mathbb{N}}
\newcommand{\CC}{\mathbb{C}}
\newcommand{\KK}{\mathbb{K}}
\newcommand{\QQ}{\mathbb{Q}}
\newcommand{\GG}{\mathbb{G}}

% --- Tekst w math mode
\newcommand{\dm}[1]{\displaystyle{#1}}  % skrót do displaystyle
\newcommand{\textq}[1]{\quad \text{#1} \quad}
\newcommand{\textqq}[1]{\qquad \text{#1} \qquad}

% --- Analiza matematyczna
\newcommand{\Dom}{\operatorname{Dom}}
\newcommand{\sgn}{\operatorname{sgn}}
\newcommand{\floor}[1]{\left\lfloor #1 \right\rfloor}
\newcommand{\ceil}[1]{\left\lceil #1 \right\rceil}
\newcommand{\inj}{\xrightarrow[]{1-1}}
\newcommand{\surj}{\xrightarrow[]{\text{na}}}
\newcommand{\bij}{\xrightarrow[\text{na}]{1-1}}
\renewcommand{\tg}{\operatorname{tg}}
\newcommand{\arctg}{\operatorname{arctg}}

\newcommand{\Sum}{\sum\limits}
\newcommand{\sumn}{\sum_{n = 0}^{\infty}}
\newcommand{\Sumn}{\sum\limits_{n = 0}^{\infty}}

\newcommand{\Lim}{\lim\limits}
\newcommand{\limn}{\lim_{n\to\infty}}
\newcommand{\Limn}{\lim\limits_{n\to\infty}}
\newcommand{\Limsup}{\limsup\limits}
\newcommand{\limsupn}{\limsup_{n\to\infty}}
\newcommand{\Limsupn}{\limsup\limits_{n\to\infty}}
\newcommand{\limxy}{\lim_{(x,y) \to (0,0)}}
\newcommand{\Limxy}{\lim\limits_{(x,y) \to (0,0)}}
\newcommand{\limuv}{\lim_{\vect{u \\ v} \to 0}}
\newcommand{\Limuv}{\lim\limits_{\vect{u \\ v} \to 0}}
\newcommand{\toto}{\rightrightarrows} % jednostajna zbieżność

\newcommand{\dx}{\ \textnormal{d}x}
\newcommand{\dy}{\ \textnormal{d}y}
\newcommand{\dz}{\ \textnormal{d}z}
\newcommand{\dt}{\ \textnormal{d}t}
\newcommand{\du}{\ \textnormal{d}u}
\newcommand{\dfdx}{\frac{\partial f}{\partial x}}
\newcommand{\dfdy}{\frac{\partial f}{\partial y}}

% --- Algebra liniowa
\renewcommand{\Re}{\operatorname{Re}}
\renewcommand{\Im}{\operatorname{Im}}
\newcommand{\compconj}[1]{\overline{#1}} % sprzężenie

\newcommand{\spn}{\operatorname{span}}
\newcommand{\vect}[1]{\left[\begin{smallmatrix} #1 \end{smallmatrix}\right]}
\newcommand{\Vect}[1]{\begin{bmatrix} #1 \end{bmatrix}}
\newcommand{\twovect}[2]{\genfrac{[}{]}{0pt}{}{#1}{#2}}

\newcommand{\im}{\operatorname{im}}
\newcommand{\rank}{\operatorname{rank}}
\renewcommand{\det}{\operatorname{det}}

\newcommand{\norm}[1]{\left\| #1 \right\|}
\newcommand{\emptynorm}{\norm{\hspace{0.2em} \cdot \hspace{0.2em} }}
\newcommand{\spr}[1]{\langle #1 \rangle} % iloczyn skalarny

% --- Teoria mnogości
\newcommand{\card}[1]{\overline{\overline{#1}}} % moc zbioru
\newcommand{\alef}{\aleph_0}
\newcommand{\cont}{\mathfrak{C}}
\newcommand{\pot}[1]{\mathcal{P}(#1)}     % zbiór potęgowy
\def\pusty{\varnothing}     % zbiór pusty
\def\puste{\varnothing}     % zbiór pusty v2

% --- Matematyka dyskretna
\newcommand{\pov}[1]{{^{\overline{#1}}}}    % górna silnia
\newcommand{\pun}[1]{{^{\underline{#1}}}}   % dolna silnia 
\newcommand{\firststir}[2]{\genfrac{[}{]}{0pt}{}{#1}{#2}} % liczba Stirlinga pierwszego rodzaju
\newcommand{\secondstir}[2]{\genfrac{\{}{\}}{0pt}{}{#1}{#2}} % liczba Stirlinga drugiego rodzaju

% --- Rachunek prawdopodobieństwa
\newcommand{\rpBinom}{\operatorname{Binom}}
\newcommand{\rpGeom}{\operatorname{Geom}}
\newcommand{\rpPois}{\operatorname{Pois}}
\newcommand{\rpUnif}{\operatorname{Unif}}
\newcommand{\rpExp}{\operatorname{Exp}}
\newcommand{\rpN}{\operatorname{N}}
\newcommand{\E}{\mathrm{E}}
\newcommand{\Var}{\operatorname{Var}}

% --- Kolory
\newcommand{\purple}[1]{\textcolor{purple}{#1}}
\newcommand{\teal}[1]{\textcolor{teal}{#1}}
\newcommand{\gray}[1]{\textcolor{gray}{#1}}
\newcommand{\grey}[1]{\textcolor{gray}{#1}}
\newcommand{\red}[1]{\textcolor{red}{#1}}
\newcommand{\orange}[1]{\textcolor{orange}{#1}}
\newcommand{\white}[1]{\textcolor{white}{#1}}
\newcommand{\black}[1]{\textcolor{black}{#1}}

% --- Wykresy
\usepackage{tikz}
\usepackage{pgfplots}
\usetikzlibrary{patterns}
\pgfplotsset{
    width = 7.6cm,
    compat = 1.9,
    xlabel = $x$,
    ylabel = $y$,
    axis lines = center
}

% --- Kod
\newcommand{\complexity}[4]{\textbf{Złożoność obliczeniowa:} $\begin{cases} \text{optymistyczna: } \purple{\Omega(#1)} \\ \text{oczekiwana: } \purple{\Theta(#2)} \\ \text{pesymistyczna: } \purple{O(#3)} \end{cases}$ \textbf{oraz pamięciowa:} $\teal{O(#4)}$}

\newminted[plain]{text}{
    autogobble,
    mathescape,
    frame = leftline,
    framerule = 1pt,
    framesep = 8pt
}

\newminted[cpp]{cpp}{
    autogobble,
    mathescape,
    frame = leftline,
    framerule = 1pt,
    framesep = 8pt
}
\newmintinline{cpp}{}

\newminted[java]{java}{
    autogobble,
    mathescape,
    frame = leftline,
    framerule = 1pt,
    framesep = 8pt
}
\newmintinline{java}{}

\newminted[sql]{sql}{
    autogobble,
    mathescape,
    frame = leftline,
    framerule = 1pt,
    framesep = 8pt
}
\newmintinline{sql}{}

\newminted[gas]{gas}{
    autogobble,
    mathescape,
    frame = leftline,
    framerule = 1pt,
    framesep = 8pt
}
\newmintinline{gas}{}

\newminted[js]{js}{
    autogobble,
    mathescape,
    frame = leftline,
    framerule = 1pt,
    framesep = 8pt
}
\newmintinline{js}{}

\newminted[html]{html}{
    autogobble,
    mathescape,
    frame = leftline,
    framerule = 1pt,
    framesep = 8pt
}
\newmintinline{html}{}

\newminted[css]{css}{
    autogobble,
    mathescape,
    frame = leftline,
    framerule = 1pt,
    framesep = 8pt
}
\newmintinline{css}{}

% --- Zadania, rozwiązania i odpowiedzi
\newcounter{problem}
\counterwithin{problem}{chapter}
\newcounter{solution}
\counterwithin{solution}{chapter}

\newenvironment{problems}{
    \begin{problemsf}
    \begin{enumerate}
}{
    \end{enumerate}
    \end{problemsf}
}

\newenvironment{solutions}{
    \section{Rozwiązania}
    \begin{solutionsf}
    \begin{enumerate}
}{
    \end{enumerate}
    \end{solutionsf}
}

\newcommand{\prob}{
    \stepcounter{problem}
    \item[\hypertarget{sol\theproblem}{} \hyperlink{prob\theproblem}{\black{\sf\textbf{\theproblem.}}}]
}

\newcommand{\sol}{
    \stepcounter{solution}
    \item[\hypertarget{prob\thesolution}{}\hyperlink{sol\thesolution}{\black{\sf\textbf{\thesolution.}}}]
}

\newcommand{\answers}[3]{
    \begin{itemize}[leftmargin = 1.5cm]
        \item[\fbox{\phantom{\texttt{TAK}}} \ \textbf{A.}] #1
        \item[\fbox{\phantom{\texttt{TAK}}} \ \textbf{B.}] #2
        \item[\fbox{\phantom{\texttt{TAK}}} \ \textbf{C.}] #3
    \end{itemize}
}

\newcommand{\answerss}[6]{
    \begin{itemize}[leftmargin = 1.5cm]
        \item[\fbox{\texttt{#4}} \ \textbf{A.}] #1
        \item[\fbox{\texttt{#5}} \ \textbf{B.}] #2
        \item[\fbox{\texttt{#6}} \ \textbf{C.}] #3
    \end{itemize}
}

% --- Lematy, Twierdzenia
\newtheorem{lemma}{Lemat}
\newtheorem{theorem}{Twierdzenie}