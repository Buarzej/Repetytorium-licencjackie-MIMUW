\chapter*{Od autorów}

Obszerna treść tej publikacji ma za zadanie dobrze przygotować Cię do egzaminu licencjackiego na MIM-ie. Można ją traktować jako coś na kształt repetytorium maturalnego, bo to właśnie na nich wzorowaliśmy się najbardziej, pisząc i składając ten dokument. W związku z tym wewnątrz nie zabraknie tak przyziemnych i~błahych elementów, jak kolorowe ramki i pogrubienia ważnych pojęć. Oraz, co najważniejsze, mnóstwo przykładów! Zależało nam, aby wszystko było jak najbardziej uporządkowane i jasne, żeby przyspieszyć i~uatrakcyjnić proces powtórki.

Ponieważ całość jest rozwijana przez dobrowolne kontrybucje, mogą pojawić się tu różne błędy: zarówno literówki, jak i poważniejsze usterki. Wszelkie uwagi (w tym sugestie, co poprawić, gdzie warto napisać dokładniejsze wyjaśnienie, gdzie umieścić rysunek itp.) są mile widziane i z całego serca będziemy za nie wdzięczni! Śmiało zgłaszajcie je w zakładce \textit{Issues} \href{https://github.com/Buarzej/Repetytorium-licencjackie-MIMUW}{w tym repozytorium na GitHubie}. Niech ta publikacja stanowi nasze wspólne dobro i służy najbliższym pokoleniom!

Przy okazji, jeśli chcecie przyczynić się do (nawet drobnego) rozwoju repetytorium, więcej informacji znajdziecie w pliku \textit{readme} we wcześniej wspomnianym repozytorium.

\subsection{Formuła egzaminu licencjackiego}

Egzamin licencjacki jest zwieńczeniem 3-letnich studiów licencjackich, a dodatkowo stanowi jednocześnie egzamin wstępny na studia magisterskie z informatyki. Składa się z \textbf{30 zadań}, na rozwiązanie których jest przeznaczone \textbf{150 minut}.

Pytania na egzaminie dotyczą wybranych zagadnień z przedmiotów obowiązkowych, do których tutaj odnosić się będziemy jako ,,podstawa programowa''. Można ją zobaczyć na \href{https://www.mimuw.edu.pl/pytania-na-egzamin-licencjacki#informatyka}{stronie wydziału}, ale wszystkie jej fragmenty znajdują się także w tym dokumencie przy odpowiednich rozdziałach.

W każdym spośród 30 zadań podane są trzy warianty: \textbf{A}, \textbf{B} i \textbf{C}. W kratce przy każdym z wariantów należy odpowiedzieć, czy jest on prawdziwy, wpisując drukowanymi literami \texttt{TAK} albo \texttt{NIE}. W przypadku omyłkowego wpisu kratkę należy przekreślić i napisać jedno z tych słów po jej lewej stronie.

Oto przykład poprawnie rozwiązanego zadania:
\begin{enumerate}
    \item[\sf\textbf{1.}] Każda liczba całkowita postaci $10^n - 1$, gdzie $n$ jest całkowite i dodatnie
    \answerss{dzieli się przez 9}{jest pierwsza}{jest nieparzysta}{TAK}{NIE}{TAK}
\end{enumerate}

Na teście nie pojawiają się żadne zadania innego typu niż wyżej pokazane.

\subsection{Zasady punktowania}

Zdający zdobywa punkty ,,duże'' (od 0 do 30) oraz ,,małe'' (od 0 do 90):
\begin{itemize}
    \item jeden punkt ,,duży'' jest przyznawany za zadanie, w którym poprawnie wskazana jest prawdziwość albo fałszywość \textbf{wszystkich trzech wariantów};
    \item jeden punkt ,,mały'' jest przyznawany za każde poprawne wskazanie prawdziwości albo fałszu \textbf{pojedynczego wariantu odpowiedzi}.
\end{itemize}
Oznacza to, że całkowicie poprawnie rozwiązane zadanie zwiększa nasz wynik o trzy ,,małe'' punkty i jeden ,,duży'' punkt.

Ostatecznym wynikiem egzaminu jest liczba $D + 0.01m$, gdzie $D$ oznacza liczbę ,,dużych'', a $m$ liczbę ,,małych'' punktów. Na przykład, wynik $5.50$ oznacza, że kandydat poprawnie wskazał w całym teście prawdziwość albo fałszywość łącznie 50 wariantów odpowiedzi, w tym każdego z trzech wariantów dla pewnych pięciu zadań. Z egzaminu nie można zdobyć więcej niż 30 punktów (wszystkie wyższe wyniki są obcinane w dół do 30).

Zasadniczą rolę w ostatecznym wyniku testu mają więc punkty ,,duże''. Punkty ,,małe'' służą jedynie rozstrzyganiu sytuacji, w których wielu kandydatów dostało tyle samo ,,dużych'' punktów.

\subsection{Układ repetytorium}

Publikacja podzielona jest na rozdziały względem przedmiotów obowiązkowych, a każdy z nich na podrozdziały o różnej tematyce materiału. Podział materiału jest ściśle skorelowany z podstawą programową egzaminu, która jest zaprezentowana na pierwszej stronie każdego rozdziału, oraz z zadaniami występującymi na poprzednich egzaminach.

Podrozdziały zaczynają się od \textbf{teoretycznego wstępu}, przeplatanego wieloma przykładami oraz okazjonalnymi wstawkami {\sffamily\bfseries To było na egzaminie}, zawierającymi najbardziej reprezentatywne pytania z archiwalnych egzaminów wraz ze sposobem ich rozwiązania.

Po części teoretycznej każdego podrozdziału zostały umieszczone \textbf{zestawy tematycznych zadań} do samodzielnego rozwiązania, wraz z \textbf{rozwiązaniami} umieszczonymi na samym końcu rozdziału. \purple{Wskazówka: aby szybko przenieść się do rozwiązania danego zadania (w elektronicznej wersji dokumentu), wystarczy kliknąć w jego numer!}

\bigskip
Na sam koniec, chcielibyśmy podziękować wszystkim poprzednim pokoleniom, których prace archiwizacyjne nieświadomie przyczyniły się do powstania tej publikacji (jako że jedynie niewielka część poprzednich egzaminów jest \textit{oficjalnie} udostępniana). Życzymy miłej pracy z repetytorium i smacznej kawusi!

\begin{flushright}
\textit{Autorzy}
\end{flushright}